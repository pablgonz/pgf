% Copyright 2019 by Mark Wibrow
%
% This file may be distributed and/or modified
%
% 1. under the LaTeX Project Public License and/or
% 2. under the GNU Public License.
%
% See the file doc/generic/pgf/licenses/LICENSE for more details.

% This file defines the pseudorandom numbers.
%
% Version 1.414213 29/9/2007

% Pseudo-random number generation.
%
% See:
% \book@{pressetal1992,
%    author    = {William H. Press and Brian P. Flannery and Saul A.
%                 Teukolsky and William T. Vetterling},
%    title     = {Numerical Recipies in C},
%    edition   = {Second},
%    publisher = {Cambridge University Press}
% }
%
% See also, the documentation for the lcg package by Erich Janka:
% (http://www.ctan.org/tex-archive/macros/latex/contrib/lcg/lcg.pdf)
%
\def\pgfmath@rnd@m{2147483647}% LaTeX Maximum.

\begingroup
\c@pgfmath@counta=\time%
\multiply\c@pgfmath@counta by\year%
\ifnum\c@pgfmath@counta=0
  \c@pgfmath@counta=1
\fi
\xdef\pgfmath@rnd@z{\the\c@pgfmath@counta}% The seed.
\endgroup

% \pgfmathsetseed
%
% Explicitly set the seed for the generator
%
% #1 -> the new seed.
%
\def\pgfmathsetseed#1{%
    % Attention pgflibraryluamath.code.tex relies on the fact that
    % this method also modifies \pgfmathresult:
    \pgfmathparse{#1}\expandafter\pgfmath@setseed\pgfmathresult.0\pgfmath@stop
}%
\def\pgfmath@setseed#1.#2\pgfmath@stop{\def\pgfmath@rnd@z{#1}}%

% Alternative parameters - see Press et al (1992) p278-279,
% for discussion.
%
% a=16807 q=127773 r=2836
% a=48271 q=4488   r=3399
%
\def\pgfmath@rnd@a{69621}
\def\pgfmath@rnd@r{23902}
\def\pgfmath@rnd@q{30845}

% \pgfmathgeneratepseudorandomnumber
%
% A linear congruency generator for generating
% pseudo-random numbers. Generates numbers in
% the range 1 - 2^31-1.
%
\def\pgfmathgeneratepseudorandomnumber{%
    \begingroup%
        \c@pgfmath@counta=\pgfmath@rnd@z%
        \c@pgfmath@countb=\pgfmath@rnd@z%
        \c@pgfmath@countc=\pgfmath@rnd@q%
        \divide\c@pgfmath@counta by\c@pgfmath@countc%
        \multiply\c@pgfmath@counta by-\c@pgfmath@countc%
        \advance\c@pgfmath@counta by\c@pgfmath@countb
        \c@pgfmath@countc=\pgfmath@rnd@a%
        \multiply\c@pgfmath@counta by\c@pgfmath@countc%
        \c@pgfmath@countc=\pgfmath@rnd@q%
        \divide\c@pgfmath@countb by\c@pgfmath@countc%
        \c@pgfmath@countc=\pgfmath@rnd@r%
        \multiply\c@pgfmath@countb by\c@pgfmath@countc%
        \advance\c@pgfmath@counta by-\c@pgfmath@countb%
        \ifnum\c@pgfmath@counta<0\relax%
            \c@pgfmath@countb=\pgfmath@rnd@m%
            \advance\c@pgfmath@counta by\c@pgfmath@countb%
        \fi%
        \xdef\pgfmath@rnd@z{\the\c@pgfmath@counta}%
    \endgroup%
    \edef\pgfmathresult{\pgfmath@rnd@z}%
}

% \pgfmathrnd
%
% Generates a pseudo-random number between 0 and 1.
%
\pgfmathdeclarefunction{rnd}{0}{%
    \begingroup%
        \pgfmathgeneratepseudorandomnumber%
        \c@pgfmath@counta\pgfmathresult%
        \c@pgfmath@countb\c@pgfmath@counta%
        \divide\c@pgfmath@countb100001\relax% To get one.
        \multiply\c@pgfmath@countb-100001\relax%
        \advance\c@pgfmath@countb\c@pgfmath@counta%
        \advance\c@pgfmath@countb1000000\relax%
        \expandafter\pgfmathrnd@@\the\c@pgfmath@countb\pgfmath@%
        \pgfmath@returnone\pgfmath@x%
    \endgroup%
}%

\def\pgfmathrnd@@#1#2#3\pgfmath@{%
    \edef\pgfmath@temp{#2.#3}%
    \pgfmath@x=\pgfmath@temp pt\relax%
}%

% \pgfmathrand
%
% Generates a pseudo-random number between -1 and 1.
%
\pgfmathdeclarefunction{rand}{0}{%
    \begingroup%
        \pgfmathgeneratepseudorandomnumber%
        \c@pgfmath@counta\pgfmathresult%
        \c@pgfmath@countb\c@pgfmath@counta%
        \divide\c@pgfmath@countb200001\relax%
        \multiply\c@pgfmath@countb-200001\relax%
        \advance\c@pgfmath@countb\c@pgfmath@counta%
        \advance\c@pgfmath@countb-100000\relax%
        \ifnum\c@pgfmath@countb<0\relax%
            \advance\c@pgfmath@countb-1000000\relax%
        \else%
            \advance\c@pgfmath@countb1000000\relax%
        \fi%
        \expandafter\pgfmathrand@@\the\c@pgfmath@countb\pgfmath@%
        \pgfmath@returnone\pgfmath@x%
    \endgroup%
}%

\def\pgfmathrand@@#1#2#3#4\pgfmath@{%
    \ifx#1-%
        \edef\pgfmath@temp{-#3.#4}%
    \else%
        \edef\pgfmath@temp{#2.#3#4}%
    \fi%
    \pgfmath@x=\pgfmath@temp pt\relax%
}%

\pgfmathdeclarefunction{random}{...}{%
    \begingroup%
        \def\pgfmath@temp{#1}%
        \ifx\pgfmath@temp\pgfmath@empty%
            \pgfmathrnd@%
        \else%
            \expandafter\pgfmathrandom@@#1\pgfmath@stop%
        \fi%
        \pgfmath@smuggleone\pgfmathresult%
    \endgroup}

\def\pgfmathrandom@@{\futurelet\pgfmath@token\pgfmathrandom@@@}
\def\pgfmathrandom@@@{%
    \ifx\pgfmath@token\bgroup%
        \expandafter\pgfmath@random@@@@%
    \else%
        \expandafter\pgfmath@random@@@@@%
    \fi%
}
\def\pgfmath@random@@@@@#1\pgfmath@stop{%
    \pgfmathrandominteger\pgfmathresult{1}{#1}%
}
\def\pgfmath@random@@@@#1#2\pgfmath@stop{%
    \pgfmathrandominteger\pgfmathresult{#1}{#2}%
}

% \pgfmathrandominteger
%
% Return a 'randomly' selected integer in the range #2 - #3 (inclusive).
%
% #1 - a macro to store the integer (not a count register).
% #2 - the lower limit of the range.
% #3 - the upper limit of the range.
%
\def\pgfmathrandominteger#1#2#3{%
    \begingroup%
        \pgfmathsetcount\c@pgfmath@counta{#2}%
        \pgfmathsetcount\c@pgfmath@countb{#3}%
        \c@pgfmath@countc\c@pgfmath@countb%
        % OK. Maybe #2 > #3.
        \ifnum\c@pgfmath@counta>\c@pgfmath@countb\relax%
            \c@pgfmath@countc\c@pgfmath@countb%
            \c@pgfmath@countb\c@pgfmath@counta%
            \c@pgfmath@counta\c@pgfmath@countc%
        \fi%
        \advance\c@pgfmath@countc1\relax%
        \advance\c@pgfmath@countc-\c@pgfmath@counta%
        \pgfmathgeneratepseudorandomnumber%
        \c@pgfmath@counta\pgfmathresult\relax%
        \c@pgfmath@countb\c@pgfmath@counta%
        \divide\c@pgfmath@countb\c@pgfmath@countc%
        \multiply\c@pgfmath@countb-\c@pgfmath@countc%
        \advance\c@pgfmath@counta\c@pgfmath@countb%
        \pgfmathaddtocount\c@pgfmath@counta{#2}%
        \edef\pgfmathresult{\the\c@pgfmath@counta}%
        \pgfmath@smuggleone{\pgfmathresult}%
    \endgroup%
    \edef#1{\pgfmathresult}%
}

% \pgfmathdeclarerandomlist
%
% Create a list to be used with \pgfmathrandomelement.
%
% #1 - the name of the list
% #2 - a list of comma separated elements.
%
\def\pgfmathdeclarerandomlist#1#2{%
    \def\pgfmath@randomlistname{#1}%
    \begingroup%
        \c@pgfmath@counta=1\relax%
        % {} is a possible random element so (locally)
        % redefine \pgfmath@empty.
        \def\pgfmath@empty{pgfmath@empty}%
        \expandafter\pgfmath@scanrandomlist#2{pgfmath@empty}}
\def\pgfmath@scanrandomlist#1{%
    \def\pgfmath@scanneditem{#1}%
    \ifx\pgfmath@scanneditem\pgfmath@empty%
        \advance\c@pgfmath@counta-1\relax%
        \expandafter\xdef\csname pgfmath@randomlist@\pgfmath@randomlistname\endcsname{\the\c@pgfmath@counta}%
        \endgroup%
    \else%
        \expandafter\gdef\csname pgfmath@randomlist@\pgfmath@randomlistname @\the\c@pgfmath@counta\endcsname{#1}%
        \advance\c@pgfmath@counta1\relax%
        \expandafter\pgfmath@scanrandomlist%
    \fi}

% \pgfmathrandomitem
%
% Return a 'randomly' selected element from a list.
%
% #1 - a macro to store the item.
% #2 - the name of the list.
%
\def\pgfmathrandomitem#1#2{%
    \pgfmath@ifundefined{pgfmath@randomlist@#2}{\pgfmath@error{Unknown random list `#2'}}{%
        \edef\pgfmath@randomlistlength{\csname pgfmath@randomlist@#2\endcsname}%
        \pgfmathrandominteger{\pgfmath@randomtemp}{1}{\pgfmath@randomlistlength}%
        \def#1{\csname pgfmath@randomlist@#2@\pgfmath@randomtemp\endcsname}}}
